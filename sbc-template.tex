\documentclass[12pt]{article}

\usepackage{sbc-template}
\usepackage{graphicx,url}
\usepackage[utf8]{inputenc}
\usepackage[brazil]{babel}
\usepackage[latin1]{inputenc}
\usepackage{amssymb}
\usepackage{csquotes}


\sloppy

\title{Digitovisky: um jogo educacional para ensino de digitação}

\author{
\parbox{\textwidth}{
\centering
Maria Eduarda de Pinho Braga\inst{1},
Eduardo Antunes dos Santos Vieira\inst{1},
Amanda Caroline Melo Assunção\inst{1},
Caio Menezes Oliveira\inst{1},\\
Ingrid Luíza Rodrigues Silva\inst{1},
Manoel Augusto Alves da Costa\inst{1},
Gabriel Marcus de Oliveira Felix\inst{1}
}
}

\address{
\inst{1} Universidade Federal de Viçosa (UFV)\\
Florestal - MG - Brasil\\
\email{
maria.e.braga@ufv.br,
eduardo.a.vieira@ufv.br,
amanda.c.assuncao@ufv.br,
caio.menezes@ufv.br,
ingrid.r.silva@ufv.br,
manoel.costa@ufv.br,
gabriel.marcus@ufv.br
}
}

\begin{document}

\maketitle

\begin{abstract}
  This meta-paper describes the style to be used in articles and short papers
  for SBC conferences. For papers in English, you should add just an abstract
  while for the papers in Portuguese, we also ask for an abstract in
  Portuguese (``resumo''). In both cases, abstracts should not have more than
  10 lines and must be in the first page of the paper.
\end{abstract}

\begin{resumo}
  Este meta-artigo descreve o estilo a ser usado na confecção de artigos e
  resumos de artigos para publicação nos anais das conferências organizadas
  pela SBC. É solicitada a escrita de resumo e abstract apenas para os artigos
  escritos em português. Artigos em inglês deverão apresentar apenas abstract.
  Nos dois casos, o autor deve tomar cuidado para que o resumo (e o abstract)
  não ultrapassem 10 linhas cada, sendo que ambos devem estar na primeira
  página do artigo.
\end{resumo}


\section{Introdução}

All full papers and posters (short papers) submitted to some SBC conference,
including any supporting documents, should be written in English or in
Portuguese. The format paper should be A4 with single column, 3.5 cm for upper
margin, 2.5 cm for bottom margin and 3.0 cm for lateral margins, without
headers or footers. The main font must be Times, 12 point nominal size, with 6
points of space before each paragraph. Page numbers must be suppressed.

Full papers must respect the page limits defined by the conference.
Conferences that publish just abstracts ask for \textbf{one}-page texts.

\section{Trabalhos relacionados} \label{sec:firstpage}

O ensino do pensamento computacional se mostra um desafio nas escolas
brasileiras <citação>. Diante disso, a literatura reconhece a eficácia dos
\textit{Serious Games (SG)} e da Aprendizagem Baseada em Jogos (\textit{Game
based learning - GBL}) como estratégias ativas para aumentar o engajamento, a
motivação e a compreensão, convertendo conceitos abstratos em experiências
práticas \cite{Luccas2025SeriousGames}. O desenvolvimento de um sistema web para
ensinar digitação a crianças do 4º ano, como o Digitovisky, é precedido por
trabalhos que também exploram o ensino do pensamento computacional para a
Educação Básica.

Na área de Informática na Educação, pesquisadores têm explorado bastante o uso
de SGs e GBL, aplicando essas ferramentas em vários conceitos e para públicos
bem diferentes. Por exemplo, quando olhamos para o design e a avaliação de
jogos, o trabalho \cite{Araujo2025CodeBo} apresenta um rigoroso processo de
desenvolvimento e avaliação de um puzzle game. Este estudo foca em conceitos de
Ensino Superior, mas estabelece uma base metodológica importante para a
avaliação de engajamento e aprendizado.

A fronteira de atuação dos jogos educacionais se estende ao uso de tecnologias
imersivas, como detalhado no \cite{Lima2022AmongNET}. Este estudo destaca a
adaptabilidade das ferramentas lúdicas ao contexto de ensino remoto e o
potencial de engajamento da Realidade Aumentada.

Em \cite{Dutra2022JogoDI}, os autores apresentam o jogo digital educacional
\enquote{Pensar e Lavar}, cujo objetivo é ensinar crianças sobre a atividade
coditiana de lavagem de roupas, com foco especial em crianças portadoras de
deficiências intelectuais. O desenvolvimento do \textit{software} foi validado
por especialistas em tais deficiências, e o produto final apresenta uma série de
recursos de acessibilidade muito eficazes na aplicação do jogo no contexto ao
qual se destina. No entanto, não foi validado com seu público-alvo propriamente
dito, uma limitação que o presente trabalho não exibe.

\begin{table}[h!]
\centering
\begin{tabular}{|p{2.2cm}|c|c|c|c|c|}
\hline
\textbf{Referência} & \begin{tabular}[c]{@{}c@{}}\textbf{Público-Alvo}\\ \textbf{Foco 4º EF}\end{tabular} &
\begin{tabular}[c]{@{}c@{}}\textbf{Digital}\end{tabular} &
\begin{tabular}[c]{@{}c@{}}\textbf{Foco na}\\ \textbf{BNCC}\\ \textbf{(EB)}\end{tabular} &
\begin{tabular}[c]{@{}c@{}}\textbf{Método de}\\ \textbf{Avaliação}\end{tabular} &
\begin{tabular}[c]{@{}c@{}}\textbf{Conteúdo}\\ \textbf{Avançado}\end{tabular} \\
\hline
\hline
\textbf{Digitóvisky} & \textbf{$\checkmark$} & \textbf{$\checkmark$} & \textbf{$\checkmark$} & \textbf{MEEGA+ (Adaptado)} & \textbf{×} \\
\hline
{[}Rosa et al.{]} & $\checkmark$ & × & $\checkmark$ & Testes (Pré/Pós) & × \\
\hline
{[}Luccas et al.{]} & × & $\checkmark$ & × & Likert (5 pontos) & $\checkmark$ \\
\hline
{[}Dôndici et al.{]} & × & $\checkmark$ & × & MSL (Diversos) & $\checkmark$ \\
\hline
{[}Mourão et al.{]} & × & × & × & Observação, Likert & × \\
\hline
{[}Guarda et al.{]} & × & × & × & Comparativo de Notas & × \\
\hline
{[}Dutra et al.{]} & $\checkmark$ & $\checkmark$ & $\checkmark$ & Qualitativo, Grupo Focal & × \\
\hline
{[}Ramos et al.{]} & × & $\checkmark$ & × & Likert (5 pontos) & $\checkmark$ \\
\hline
{[}Oliveira et al.{]} & × & $\checkmark$ & × & Observação & × \\
\hline
{[}Araujo e Silva{]} & × & $\checkmark$ & × & E-GameFlow, Logs & $\checkmark$ \\
\hline
{[}Braga e Sasaki{]} & × & × & $\checkmark$ & Testes de Jogabilidade & × \\
\hline
{[}Lima et al.{]} & × & $\checkmark$ & × & MEEGA+ (Adaptado) & $\checkmark$ \\
\hline
{[}Schoeffel, 2021{]} & × & × & × & MEEGA (Pós-teste) & $\checkmark$ \\
\hline
\end{tabular}
\caption{Tabela 1. Comparativo de Trabalhos Relacionados e o Sistema Proposto.}
\end{table}



\section{Sections and Paragraphs}

Section titles must be in boldface, 13pt, flush left. There should be an extra
12 pt of space before each title. Section numbering is optional. The first
paragraph of each section should not be indented, while the first lines of
subsequent paragraphs should be indented by 1.27 cm.

\subsection{Subsections}

The subsection titles must be in boldface, 12pt, flush left.

\section{Figures and Captions}\label{sec:figs}


Figure and table captions should be centered if less than one line
(Figure~\ref{fig:exampleFig1}), otherwise justified and indented by 0.8cm on
both margins, as shown in Figure~\ref{fig:exampleFig2}. The caption font must
be Helvetica, 10 point, boldface, with 6 points of space before and after each
caption.

\begin{figure}[ht]
\centering
\includegraphics[width=.5\textwidth]{fig1.jpg}
\caption{A typical figure}
\label{fig:exampleFig1}
\end{figure}

\begin{figure}[ht]
\centering
\includegraphics[width=.3\textwidth]{fig2.jpg}
\caption{This figure is an example of a figure caption taking more than one
  line and justified considering margins mentioned in Section~\ref{sec:figs}.}
\label{fig:exampleFig2}
\end{figure}

In tables, try to avoid the use of colored or shaded backgrounds, and avoid
thick, doubled, or unnecessary framing lines. When reporting empirical data,
do not use more decimal digits than warranted by their precision and
reproducibility. Table caption must be placed before the table (see Table 1)
and the font used must also be Helvetica, 10 point, boldface, with 6 points of
space before and after each caption.

\begin{table}[ht]
\centering
\caption{Variables to be considered on the evaluation of interaction
  techniques}
\label{tab:exTable1}
\includegraphics[width=.7\textwidth]{table.jpg}
\end{table}

\section{Images}

All images and illustrations should be in black-and-white, or gray tones,
excepting for the papers that will be electronically available (on CD-ROMs,
internet, etc.). The image resolution on paper should be about 600 dpi for
black-and-white images, and 150-300 dpi for grayscale images.  Do not include
images with excessive resolution, as they may take hours to print, without any
visible difference in the result.

\section{References}

Bibliographic references must be unambiguous and uniform.  We recommend giving
the author names references in brackets, e.g. \cite{knuth:84},
\cite{boulic:91}, and \cite{smith:99}.

The references must be listed using 12 point font size, with 6 points of space
before each reference. The first line of each reference should not be
indented, while the subsequent should be indented by 0.5 cm.

\bibliographystyle{sbc}
\bibliography{sbc-template}

\end{document}
% Estou colocando a linha abaixo como se ela fosse magicamente garantir que o
% texto se limitasse a 80 colunas mesmo para quem não usa o vim
% vim:set textwidth=80:
